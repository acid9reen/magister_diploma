\section*{Аннотация}
\addcontentsline{toc}{chapter}{Реферат}

В современном медицинском исследовании приоритетное значение имеет изучение
сердечно-сосудистых заболеваний, которые становятся всё более распространёнными
и смертельными по всему миру. Электрограммы, записи биоэлектрической активности
сердца, предоставляют более точную информацию о его функционировании, нежели
традиционная электрокардиография. Однако, анализ большого объёма электрограмм
вручную требует значительных усилий и подвержен ошибкам, особенно в условиях
большого разнообразия данных. Разработка инновационных методов обработки данных,
включая применение глубокого обучения, становится необходимой для автоматизации
этого процесса.

Прямые методы сегментации, хоть и просты в реализации и обработке, требуют
тщательной настройки параметров и могут быть менее устойчивы к шумам и
изменчивости данных. В то время как методы на основе глубокого обучения, хоть и
требуют больших объёмов данных и вычислительных ресурсов для обучения, способны
автоматически извлекать признаки из данных, обеспечивая высокую точность
сегментации и универсальность в применении для различных типов сигналов.

Данная работа описывает применение глубоких сверточных сетей для сегментации
электрограмм.

\section*{Abstract}
\addcontentsline{toc}{chapter}{Abstract}
