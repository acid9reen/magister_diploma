\section*{Аннотация}
\addcontentsline{toc}{chapter}{Аннотация}

В современном медицинском исследовании приоритетное значение имеет изучение
сердечно-сосудистых заболеваний, которые остаются первой причиной
смертности населения по всему миру. Многоканальная регистрация электрограмм
записей биоэлектрической активности сердца с поверхности эпикарда и эндокарда в
некоторых случаях является более информативным методом нежели традиционная
электрокардиография. Однако, анализ большого объёма электрограмм вручную
требует значительных усилий и подвержен ошибкам, особенно в условиях большого
разнообразия данных. Разработка инновационных методов обработки данных, включая
применение глубокого обучения, становится необходимой для автоматизации этого
процесса.

Прямые методы сегментации, хоть и просты в реализации и обработке, требуют
тщательной настройки параметров и могут быть менее устойчивы к шумам и
изменчивости данных. В то время как методы на основе глубокого обучения, хоть и
требуют больших объёмов данных и вычислительных ресурсов для обучения, способны
автоматически извлекать признаки из данных, обеспечивая высокую точность
сегментации и универсальность в применении для различных типов сигналов.

Данная работа описывает применение глубоких сверточных сетей для сегментации
электрограмм. С помощью предлагаемого метода удалось достичь прироста качества
анализа электрограммы на 11\% в полностью автоматическом режиме, то есть без
тонкой настройки предметным специалистом.
