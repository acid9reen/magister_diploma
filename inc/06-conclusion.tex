\chapter{Заключение}
\section{Выводы}

В представленной работе обсуждается методика улучшения процесса анализа
сигналов электрограммы (ЭГ) с использованием технологий глубокого обучения.
Основная цель исследования заключается в разработке метода, который
автоматически обнаруживает активации в сигналах ЭГ и предоставляет возможность
сравнения их с исходной разметкой для оценки качества работы модели.

В ходе исследования были выполнены следующие этапы:

\begin{enumerate}

    \item Обработка экспериментальных данных: Был проведен анализ и обработка
    данных, а так же создан датасет на их основе для решения поставленной
    задачи.

    \item Обучение модели: На этом этапе была разработана и обучена модель глубокого
    обучения для автоматического обнаружения активаций в сигналах ЭГ. Для обучения
    модели использовались размеченные данные, включающие в себя сигналы ЭГ и
    соответствующую разметку.

    \item Анализ результатов: После обучения модели проводился
    анализ ее результатов. Этот этап включал в себя оценку точности и полноты
    обнаружения активаций, а также сравнение результатов работы модели с исходной
    разметкой.

    \item Сравнение с оригинальной разметкой: Для оценки качества работы
    модели было выполнено сравнение ее результатов с оригинальной разметкой. Это
    позволило выявить различия между обнаруженными активациями и истинными
    активациями в сигналах ЭГ. Результаты исследования показали, что разработанный
    метод обеспечивает прирост качества работы на 10\% по сравнению с оригинальной
    программой. Это значительное улучшение производительности, особенно учитывая,
    что метод является полностью автоматическим и не требует сложной настройки.

\end{enumerate}

Таким образом, данное исследование демонстрирует значительный прогресс в
области анализа сигналов ЭГ с использованием методов глубокого обучения.
Полученные результаты могут иметь важное значение для дальнейших исследований в
области применения искусственных неронных сетей в задачах медицины.

\section{Перспективы}

В контексте данного исследования представляется ряд перспективных направлений
для дальнейших исследований и улучшения метода анализа сигналов
электрограммы (ЭГ).

\begin{enumerate}

    \item Онлайн дообучение на новых данных: Одним из перспективных подходов
    является реализация системы онлайн дообучения модели на новых данных под
    контролем специалиста. Это позволит модели адаптироваться к изменяющимся
    условиям и различным особенностям входного сигнала. Специалист может
    контролировать процесс обучения и вносить корректировки в модель в реальном
    времени на основе новых данных и наблюдений.

    \item Исследования других архитектур нейронных сетей: Помимо текущей
    архитектуры нейронной сети, имеет смысл исследовать и сравнивать
    эффективность других архитектур, таких как трансформеры или рекуррентные
    нейронные сети (RNN). Трансформеры могут быть эффективны для анализа
    последовательностей данных, таких как сигналы ЭГ, за счет своей
    способности к обработке длинных зависимостей. Рекуррентные нейронные сети
    также могут быть полезны для анализа последовательных данных, но требуют
    особого внимания к проблеме затухающего или взрывного градиента.

    \item Оценка обобщаемости модели: Важным направлением исследования является
    оценка обобщаемости модели на новых наборах данных. Это позволит
    определить, насколько хорошо модель способна работать на данных из
    различных источников.

    \item Улучшение методов обнаружения активаций: В дополнение к архитектуре
    нейронной сети, также стоит обратить внимание на методы обнаружения
    активаций в сигналах ЭГ. Можно исследовать различные подходы к определению
    порогов активации, а также улучшение алгоритмов поиска активаций для
    повышения точности и полноты обнаружения.

\end{enumerate}

Исследование этих перспективных направлений позволит улучшить эффективность и
применимость метода анализа сигналов ЭГ на практике, что в конечном итоге может
способствовать развитию диагностики и лечения различных сердечно-сосудистых
заболеваний.
