\begingroup
\begin{thebibliography}{00}
    \addcontentsline{toc}{chapter}{Литература}
    \bibitem{egm-dataset-source}
    Исходный egm\_dataset код доступен по ссылке: https://github.com/acid9reen/egm\_dataset

    \bibitem{egm-analyzer-source}
    Исходный код egm\_analyzer доступен по ссылке: https://github.com/acid9reen/egm\_analyzer

    \bibitem{heartbeat-detector-source}
    Исходный код heartbeat\_detector доступен по ссылке:\\
    https://github.com/acid9reen/heartbeat\_detector

    \bibitem{dsp-guide}
    Smith S. W. et al. The scientist and engineer's guide to digital signal processing. -- 1997.

    \bibitem{unet}
    Ronneberger O., Fischer P., Brox T. U-net: Convolutional networks for
    biomedical image segmentation //Medical image computing and
    computer-assisted intervention–MICCAI 2015: 18th international conference,
    Munich, Germany, October 5-9, 2015, proceedings, part III 18. – Springer
    International Publishing, 2015. – С. 234-241.

    \bibitem{anton}
    Антон Рыбкин, ''Тут название работы'' (2024)

     \bibitem{geron}
    Geron, Aurelien. Hands-On Machine Learning with Scikit-Learn \& TensorFlow
    : concepts, tools, and techniques to build intelligent systems. Beijing:
    O`Reilly, 2017. Text.

    \bibitem{goodfellow}
    Ian Goodfellow and Yoshua Bengio and Aaron Courville. Deep Learning. MIT
    press, 2016.\\ http://www.deeplearningbook.org

    \bibitem{prep}
    Bigdely-Shamlo N. et al. The PREP pipeline: standardized preprocessing for
    large-scale EEG analysis //Frontiers in neuroinformatics. – 2015. – Т. 9. –
    С. 16.

    \bibitem{victor}
    Moskalenko V., Zolotykh N., Osipov G. Deep learning for ECG segmentation
    //Advances in Neural Computation, Machine Learning, and Cognitive Research
    III: Selected Papers from the XXI International Conference on
    Neuroinformatics, October 7-11, 2019, Dolgoprudny, Moscow Region, Russia. –
    Springer International Publishing, 2020. – С. 246-254.

    \bibitem{adam}
    Kingma D. P., Ba J. Adam: A method for stochastic optimization //arXiv
    preprint arXiv:1412.6980. – 2014.

    \bibitem{filters}
    Kumar K. S., Yazdanpanah B., Kumar P. R. Removal of noise from
    electrocardiogram using digital FIR and IIR filters with various methods
    //2015 International conference on communications and signal processing
    (ICCSP). – IEEE, 2015. – С. 0157-0162.

    \bibitem{batchnorm}
    Ioffe S., Szegedy C. Batch normalization: Accelerating deep network
    training by reducing internal covariate shift //International conference on
    machine learning. – pmlr, 2015. – С. 448-456.

    \bibitem{deep-learning}
    LeCun Y., Bengio Y., Hinton G. Deep learning //nature. – 2015. – Т. 521. –
    №. 7553. – С. 436-444.

    \bibitem{pytorch}
    Paszke A. et al. Pytorch: An imperative style, high-performance deep
    learning library //Advances in neural information processing systems. –
    2019. – Т. 32.

    \bibitem{scipy}
    Virtanen P. et al. SciPy 1.0: fundamental algorithms for scientific
    computing in Python //Nature methods. – 2020. – Т. 17. – №. 3. – С.
    261-272.

    \bibitem{numpy}
    Harris C. R. et al. Array programming with NumPy //Nature. – 2020. – Т.
    585. – №. 7825. – С. 357-362.

    \bibitem{lutz}
    Lutz M. Learning python: Powerful object-oriented programming. – " O'Reilly
    Media, Inc.", 2013.

    \bibitem{lena}
    Kharkovskaya E. Е., Osipov G. V., Mukhina I. V. Ventricular fibrillation
    induced by 2-aminoethoxydiphenyl borate under conditions of
    hypoxia/reoxygenation //Minerva Cardioangiologica. – 2020. – Т. 68. – №. 6.
    – С. 619-628.

    \bibitem{computer-aided-detection}
    Baldazzi G. et al. Computer-aided detection of arrhythmogenic sites in
    post-ischemic ventricular tachycardia //Scientific Reports. – 2023. – Т.
    13. – №. 1. – С. 6906.
\end{thebibliography}
\endgroup
