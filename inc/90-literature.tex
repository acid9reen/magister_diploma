\begingroup
\begin{thebibliography}{00}
	\addcontentsline{toc}{chapter}{Литература}
	\bibitem{who}
	Cardiovascular diseases (CVDs). --- Текст : электронный // World Health
	Organization : [сайт]. --- URL:
	https://www.who.int/news-room/fact-sheets/detail/cardiovascular-diseases-(cvds)
	(дата обращения: 08.04.2024).

	\bibitem{victor}
	Moskalenko V., Zolotykh N., Osipov G. Deep learning for ECG segmentation
	//Advances in Neural Computation, Machine Learning, and Cognitive Research
	III: Selected Papers from the XXI International Conference on
	Neuroinformatics, October 7-11, 2019, Dolgoprudny, Moscow Region, Russia. –
	Springer International Publishing, 2020. – С. 246-254.

	\bibitem{ecg-segmentation}
	Beraza I., Romero I. Comparative study of algorithms for ECG segmentation
	//Biomedical Signal Processing and Control. – 2017. – Т. 34. – С. 166-173.

	\bibitem{euristic-1}
	Martínez A., Alcaraz R., Rieta J. J. Application of the phasor transform
	for automatic delineation of single-lead ECG fiducial points
	//Physiological measurement. – 2010. – Т. 31. – №. 11. – С. 1467.

	\bibitem{euristic-2}
	Singh Y. N., Gupta P. ECG to individual identification //2008 IEEE Second
	International Conference on Biometrics: Theory, Applications and Systems. –
	IEEE, 2008. – С. 1-8.

	\bibitem{euristic-3}
	Vázquez-Seisdedos C. R. et al. New approach for T-wave end detection on
	electrocardiogram: Performance in noisy conditions //Biomedical engineering
	online. – 2011. – Т. 10. – С. 1-11.

	\bibitem{euristic-4}
	Kang W., Byun K., Kang H. G. Detection of fiducial points in ECG waves
	using iteration based adaptive thresholds //2015 37th Annual International
	Conference of the IEEE Engineering in Medicine and Biology Society (EMBC).
	– IEEE, 2015. – С. 2721-2724.

	\bibitem{euristic-5}
	Bayasi N. et al. Adaptive technique for P and T wave delineation in
	electrocardiogram signals //2014 36th Annual International Conference of
	the IEEE Engineering in Medicine and Biology Society. – IEEE, 2014. – С.
	90-93.

	\bibitem{egm-dataset-source}
	Исходный egm\_dataset код доступен по ссылке: https://github.com/acid9reen/egm\_dataset

	\bibitem{egm-analyzer-source}
	Исходный код egm\_analyzer доступен по ссылке: https://github.com/acid9reen/egm\_analyzer

	\bibitem{heartbeat-detector-source}
	Исходный код heartbeat\_detector доступен по ссылке:\\
	https://github.com/acid9reen/heartbeat\_detector

	\bibitem{dsp-guide}
	Smith S. W. et al. The scientist and engineer's guide to digital signal processing. -- 1997.

	\bibitem{unet}
	Ronneberger O., Fischer P., Brox T. U-net: Convolutional networks for
	biomedical image segmentation //Medical image computing and
	computer-assisted intervention–MICCAI 2015: 18th international conference,
	Munich, Germany, October 5-9, 2015, proceedings, part III 18. – Springer
	International Publishing, 2015. – С. 234-241.

	\bibitem{anton}
	Рыбкин А. В. ``Тут название работы'': Дипломная работа. Ниж. гос. университет, Нижний Новгород, 2024.

	\bibitem{geron}
	Geron, Aurelien. Hands-On Machine Learning with Scikit-Learn \& TensorFlow
	: concepts, tools, and techniques to build intelligent systems. Beijing:
	O`Reilly, 2017. Text.

	\bibitem{goodfellow}
	Ian Goodfellow and Yoshua Bengio and Aaron Courville. Deep Learning. MIT
	press, 2016.\\ http://www.deeplearningbook.org

	\bibitem{prep}
	Bigdely-Shamlo N. et al. The PREP pipeline: standardized preprocessing for
	large-scale EEG analysis //Frontiers in neuroinformatics. – 2015. – Т. 9. –
	С. 16.

	\bibitem{adam}
	Kingma D. P., Ba J. Adam: A method for stochastic optimization //arXiv
	preprint arXiv:1412.6980. – 2014.

	\bibitem{filters}
	Kumar K. S., Yazdanpanah B., Kumar P. R. Removal of noise from
	electrocardiogram using digital FIR and IIR filters with various methods
	//2015 International conference on communications and signal processing
	(ICCSP). – IEEE, 2015. – С. 0157-0162.

	\bibitem{batchnorm}
	Ioffe S., Szegedy C. Batch normalization: Accelerating deep network
	training by reducing internal covariate shift //International conference on
	machine learning. – pmlr, 2015. – С. 448-456.

	\bibitem{deep-learning}
	LeCun Y., Bengio Y., Hinton G. Deep learning //nature. – 2015. – Т. 521. –
	№. 7553. – С. 436-444.

	\bibitem{pytorch}
	Paszke A. et al. Pytorch: An imperative style, high-performance deep
	learning library //Advances in neural information processing systems. –
	2019. – Т. 32.

	\bibitem{scipy}
	Virtanen P. et al. SciPy 1.0: fundamental algorithms for scientific
	computing in Python //Nature methods. – 2020. – Т. 17. – №. 3. – С.
	261-272.

	\bibitem{numpy}
	Harris C. R. et al. Array programming with NumPy //Nature. – 2020. – Т.
	585. – №. 7825. – С. 357-362.

	\bibitem{lutz}
	Lutz M. Learning python: Powerful object-oriented programming. – ``O'Reilly
	Media, Inc.'', 2013.

	\bibitem{lena}
	Kharkovskaya E. Е., Osipov G. V., Mukhina I. V. Ventricular fibrillation
	induced by 2-aminoethoxydiphenyl borate under conditions of
	hypoxia/reoxygenation //Minerva Cardioangiologica. – 2020. – Т. 68. – №. 6.
	– С. 619-628.

	\bibitem{cardio1}
	Louisse J. et al. Assessment of acute and chronic toxicity of doxorubicin
	in human induced pluripotent stem cell-derived cardiomyocytes //Toxicology
	In Vitro. – 2017. – Т. 42. – С. 182-190.

	\bibitem{cardio2}
	Kujala V. J. et al. Laminar ventricular myocardium on a microelectrode
	array-based chip //Journal of Materials Chemistry B. – 2016. – Т. 4. – №.
	20. – С. 3534-3543.

	\bibitem{computer-aided-detection}
	Baldazzi G. et al. Computer-aided detection of arrhythmogenic sites in
	post-ischemic ventricular tachycardia //Scientific Reports. – 2023. – Т.
	13. – №. 1. – С. 6906.

	\bibitem{residual}
	He K. et al. Deep residual learning for image recognition //Proceedings of
	the IEEE conference on computer vision and pattern recognition. – 2016. –
	С. 770-778.

	\bibitem{dropout-is-good}
	Hinton G. E. et al. Improving neural networks by preventing co-adaptation
	of feature detectors //arXiv preprint arXiv:1207.0580. – 2012.

	\bibitem{cnn-dropout}
	Wu H., Gu X. Towards dropout training for convolutional neural networks
	//Neural Networks. – 2015. – Т. 71. – С. 1-10.

	\bibitem{first}
	Смирнов, Р.О., Рыбкин, А.В., Котихина, Е.Е., Карчков, Д.А., Москаленко,
	В.А., Смирнов, Л.А. Анализ эпикардиальных электрограмм средствами
	искусственного интеллекта // Моделирование и экспериментальные исследования
	динамики сложных систем: сборник трудов Всероссийской научной конференции,
	16-18 ноября 2023 года. - 2023. - С. 26-28.

	\bibitem{second}
	Рыбкин, А.В., Смирнов, Р.О., Котихина, Е.Е., Карчков, Д.А., Москаленко, В.А.,
	Осипов, Г.В., Смирнов, Л.А. Анализ эпикардиальных электрограмм средствами
	искусственного интеллекта // Математическое моделирование и суперкомпьютерные
	технологии: труды XXIII Международной конференции, 13-16 ноября 2023 года,
	Нижний Новгород. — 2023. — С. 129-134.

	\bibitem{lenas-disser}
	Котихина Е.Е. Исследование пространственно-временных характеристик
	биоэлектрической активности миокарда с использованием эпикардиальных
	микроэлектродных матриц // Диссертация на соискание степени кандидата
	биологических наук, специальность 1.5.5 - физиология человека и животных,
	диплом №033290 от 5.10.2023 г. (дата защиты 27.04.2023).
	https://diss.unn.ru/1332

	\bibitem{lenas-paper}
	Kharkovskaya E.E., Osipov G.V., Mukhina I.V. Ventricular fibrillation
	induced by 2-aminoethoxydiphenyl borate under conditions of
	hypoxia/reoxygenation. Minerva Cardioangiol. 2020 Vol.68 (6):619-28

\end{thebibliography}
\endgroup
